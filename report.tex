\documentclass{article}

\usepackage{graphicx} % Required for inserting images

\usepackage{amsmath} % AMS mathematical facilities for LATEX
\usepackage{amsfonts} % TEX fonts from the American Mathematical Society
\usepackage{bbold} % A geometric sans serif blackboard bold font, for use in mathematics;

\usepackage{float} % Improved interface for floating objects

\usepackage{listings} % The package enables the user to typeset programs (programming code) within LATEX
\lstset{language=Python}
\lstset
{ %Formatting for code in Python
    basicstyle=\footnotesize,
    numbers=left,
    stepnumber=1,
    showstringspaces=false,
    tabsize=1,
    breaklines=true,
    breakatwhitespace=false,
}

\setlength{\parindent}{0pt}
\usepackage{geometry} % Flexible and complete interface to document dimensions
\usepackage{todonotes}
\geometry{hmargin=2.5cm,vmargin=2.5cm}

\title{EDIN01 Cryptography \\ Project 3}
\author{Maxime Pakula, Sofia Boselli Graf}

\begin{document}

\maketitle

\tableofcontents

\newpage

\section{Correlation Attacks}
\subsection{Exercise 1}

In this exercise the task is to find the key K which results in the keystream:

$$1001000110011110011001100111000011110110101011101110000111001011010100010110000000111001011011001$$
$$000011000111000111010110010101100101001111110111111000010001011110010011111111101001110101100101$$

To this end a correlation attack is coded as follows. 
The key must be obtained for the three different LFSR in the same way. This means the code will repeat the same steps 3 times. 

Firstly, the C equation is given in an array, for example, for L1 it is defined:
\begin{verbatim}
    C1 = [1,0,1,1,0,0,1,1,0,1,0,1,1]
\end{verbatim}

This array, the desired stream and the length N of this stream are given to a function \textbf{\textit{findKeyLFSR}} defined as:

\begin{verbatim}
    def findKeyLFSR(C,N,stream):
        # C is a list of the coefficient of the LFSR
        # N is the number of output desired
        # stream is the output stream that we use to compute correlation
        L = len(C)
        p_star = []
        for i in range(1,2**L):
            init = [int(x) for x in str(bin(i))[2:]]
            while len(init) < L:
                init.insert(0,0)
            u = shiftRegister(C,N,init)
            p_star.append(calculatePStar(u,stream,N))
        bestKeyIndex = p_star.index(max(p_star))
        bestKeyInt = bestKeyIndex + 1
        return bestKeyInt
\end{verbatim}

The function goes through every possible initial state of the LFSR and provides the result of length N when passed through the shift register in \textbf{\textit{shiftRegister}}. This is:

\begin{verbatim}
    def shiftRegister(C, N, init):
        # C is a list of the coefficients of the LFSR
        # N is the number of output desired
        # init is a list of the initial state of the LFSR
        out = []
        current = init.copy()
        while len(out) < N:
            add = 0
            for coeff_id,coeff in enumerate(C):
                add += coeff*current[coeff_id]
            current.append(add%2)
            out.append(current[0])
            current.pop(0)
        return out
\end{verbatim}

Then correlation $p^*$ is calculated for each result by calculating the hamming distance between the output vector and the original stream.

\begin{verbatim}
    def calculateHamming(u,z):
        # u and z are two vectors of the same size
        hamm = 0
        for coor_id,coor in enumerate(u):
            if coor != z[coor_id]:
                hamm += 1
        return hamm

    def calculatePStar(u,z,N):
        # u and z are two vectors of the same size
        # N is the size of u and z
        hamm = calculateHamming(u,z)
        pStar = 1 - (hamm/N)
        return pStar
\end{verbatim}

Finally, key for the LFSR is the initial condition which yielded the higher correlation. This is done with the three LFSR and the final key is given. A test is also coded where the obtained key is used to calculate a stream, the test is successful if the new stream is the same as the one provided. 

\subsection{Exercise 2}

The first LFSR is of length 13 which means it has $2^{13}$ possible initial conditions, the second LFSR of length 15 has $2^{15}$ possible inital conditions and lastly, the third LFSR of length 17 has $2^{17}$ possible initial conditions. This means that an exhaustive search over the keyspace would mean trying $2^{45}$ options which would take $T2^{45}$ seconds

\todo[inline]{Not sure if this is what the exercise is about}

\section{Appendix: Code}

\begin{verbatim}
    import argparse
    from time import time
    
    def shiftRegister(C, N, init):
        # C is a list of the coefficients of the LFSR
        # N is the number of output desired
        # init is a list of the initial state of the LFSR
        out = []
        current = init.copy()
        while len(out) < N:
            add = 0
            for coeff_id,coeff in enumerate(C):
                add += coeff*current[coeff_id]
            current.append(add%2)
            out.append(current[0])
            current.pop(0)
        return out
    
    def calculateHamming(u,z):
        # u and z are two vectors of the same size
        hamm = 0
        for coor_id,coor in enumerate(u):
            if coor != z[coor_id]:
                hamm += 1
        return hamm
    
    def calculatePStar(u,z,N):
        # u and z are two vectors of the same size
        # N is the size of u and z
        hamm = calculateHamming(u,z)
        pStar = 1 - (hamm/N)
        return pStar
    
    def findKeyLFSR(C,N,stream):
        # C is a list of the coefficient of the LFSR
        # N is the number of output desired
        # stream is the output stream that we use to compute correlation
        L = len(C)
        p_star = []
        for i in range(1,2**L):
            init = [int(x) for x in str(bin(i))[2:]]
            while len(init) < L:
                init.insert(0,0)
            u = shiftRegister(C,N,init)
            p_star.append(calculatePStar(u,stream,N))
        bestKeyIndex = p_star.index(max(p_star))
        bestKeyInt = bestKeyIndex + 1
        return bestKeyInt
    
    def formatKey(L,key):
        # L is the length of the key
        # key is the integer number coresponding to the binary key
        key = [int(x) for x in str(bin(key))[2:]]
        while len(key) < L:
            key.insert(0,0)
        return key
    
    
    def checkKey(C1,C2,C3,keys,N,stream):
        # C1 is a list of the coefficients of LFSR 1
        # C2 is a list of the coefficients of LFSR 2
        # C3 is a list of the coefficients of LFSR 3
        # keys is a list of the 3 intial states of LFSR 1, 2 & 3
        # N is the length of the output stream
        # stream is the output stream
        L1 = shiftRegister(C1, N, keys[0])
        L2 = shiftRegister(C2, N, keys[1])
        L3 = shiftRegister(C3, N, keys[2])
    
        z = []
        correct = True
        for i in range(N):
            if L1[i] + L2[i] + L3[i] > 1:
                z.append(1)
            else:
                z.append(0)
            if z[i] != stream[i]:
                correct = False
                break
        return correct, z
    
    
    
    def findKey(stream, showSteps):
        N = len(stream)
        keys = []
    
        # LSFR 1
        C1 = [1,0,1,1,0,0,1,1,0,1,0,1,1]
        beginDate1 = time()
        key1 = formatKey(len(C1),findKeyLFSR(C1,N,stream)) 
        endDate1 = time()
        keys.append(key1)
        print("Broke L1! \nKey: ", key1, "\nTime required for L1: ", endDate1-beginDate1)
    
        # LSFR 2
        C2 = [1,0,1,0,1,1,0,0,1,1,0,1,0,1,0]
        beginDate2 = time()
        key2 = formatKey(len(C2),findKeyLFSR(C2,N,stream)) 
        endDate2 = time()
        keys.append(key2)
        print("Broke L2! \nKey: ", key2, "\nTime required for L2: ", endDate2-beginDate2)
    
        # LSFR 3
        C3 = [1,1,0,0,1,0,0,1,0,1,0,0,1,1,0,1,0]
        beginDate3 = time()
        key3 = formatKey(len(C3),findKeyLFSR(C3,N,stream)) 
        endDate3 = time()
        keys.append(key3)
        print("Broke L3! \nKey: ", key3, "\nTime required for L3: ", endDate3-beginDate3)
        print("Total time required: ", endDate1-beginDate1+endDate2-beginDate2+endDate3-beginDate3)
    
        print("Checking if correct:")
        correct,z = checkKey(C1,C2,C3,keys,N,stream)
        print("The key is correct? ", correct)
    
        return correct, keys
    
    
    
    if __name__ == "__main__":
    
        parser = argparse.ArgumentParser(description="Find the key!")
        parser.add_argument("--keystream", required=False, type=str)
        parser.add_argument("--showSteps", required=False, type=bool)
        args = parser.parse_args()
        showSteps = args.showSteps
    
        if args.keystream:
            stream = args.keystream
        else:
            stream = "100100011001111001100110011100001111011010101110111000011100101101010001011000...
            ....00001110010110110010000110001110001110101100101011001010011111101111110000100010111...
            ...10010011111111101001110101100101"
            stream = [int(x) for x in stream]
        
        correct, key = findKey(stream, showSteps)
        
        if correct:
            print("The key is: ", key)
        else:
            print("A key was found but it did not yield the same stream :(")
\end{verbatim}

\end{document}